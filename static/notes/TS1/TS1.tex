\documentclass[11pt,a4paper,]{article}
\usepackage{lmodern}

\usepackage{amssymb,amsmath}
\usepackage{ifxetex,ifluatex}
\usepackage{fixltx2e} % provides \textsubscript
\ifnum 0\ifxetex 1\fi\ifluatex 1\fi=0 % if pdftex
  \usepackage[T1]{fontenc}
  \usepackage[utf8]{inputenc}
\else % if luatex or xelatex
  \usepackage{unicode-math}
  \defaultfontfeatures{Ligatures=TeX,Scale=MatchLowercase}
\fi
% use upquote if available, for straight quotes in verbatim environments
\IfFileExists{upquote.sty}{\usepackage{upquote}}{}
% use microtype if available
\IfFileExists{microtype.sty}{%
\usepackage[]{microtype}
\UseMicrotypeSet[protrusion]{basicmath} % disable protrusion for tt fonts
}{}
\PassOptionsToPackage{hyphens}{url} % url is loaded by hyperref
\usepackage[unicode=true]{hyperref}
\hypersetup{
            pdftitle={Linear Time Series Analysis and Its Applications},
            pdfborder={0 0 0},
            breaklinks=true}
\urlstyle{same}  % don't use monospace font for urls
\usepackage{geometry}
\geometry{left=2.5cm,right=2.5cm,top=2.5cm,bottom=2.5cm}
\usepackage[style=authoryear-comp,]{biblatex}
\addbibresource{references.bib}
\usepackage{longtable,booktabs}
% Fix footnotes in tables (requires footnote package)
\IfFileExists{footnote.sty}{\usepackage{footnote}\makesavenoteenv{long table}}{}
\IfFileExists{parskip.sty}{%
\usepackage{parskip}
}{% else
\setlength{\parindent}{0pt}
\setlength{\parskip}{6pt plus 2pt minus 1pt}
}
\setlength{\emergencystretch}{3em}  % prevent overfull lines
\providecommand{\tightlist}{%
  \setlength{\itemsep}{0pt}\setlength{\parskip}{0pt}}
\setcounter{secnumdepth}{5}

% set default figure placement to htbp
\makeatletter
\def\fps@figure{htbp}
\makeatother


\title{Linear Time Series Analysis and Its Applications}

%% MONASH STUFF

%% CAPTIONS
\RequirePackage{caption}
\DeclareCaptionStyle{italic}[justification=centering]
 {labelfont={bf},textfont={it},labelsep=colon}
\captionsetup[figure]{style=italic,format=hang,singlelinecheck=true}
\captionsetup[table]{style=italic,format=hang,singlelinecheck=true}

%% FONT
\RequirePackage{bera}
\RequirePackage{mathpazo}

%% HEADERS AND FOOTERS
\RequirePackage{fancyhdr}
\pagestyle{fancy}
\rfoot{\Large\sffamily\raisebox{-0.1cm}{\textbf{\thepage}}}
\makeatletter
\lhead{\textsf{\expandafter{\@title}}}
\makeatother
\rhead{}
\cfoot{}
\setlength{\headheight}{15pt}
\renewcommand{\headrulewidth}{0.4pt}
\renewcommand{\footrulewidth}{0.4pt}
\fancypagestyle{plain}{%
\fancyhf{} % clear all header and footer fields
\fancyfoot[C]{\sffamily\thepage} % except the center
\renewcommand{\headrulewidth}{0pt}
\renewcommand{\footrulewidth}{0pt}}

%% MATHS
\RequirePackage{bm,amsmath}
\allowdisplaybreaks

%% GRAPHICS
\RequirePackage{graphicx}
\setcounter{topnumber}{2}
\setcounter{bottomnumber}{2}
\setcounter{totalnumber}{4}
\renewcommand{\topfraction}{0.85}
\renewcommand{\bottomfraction}{0.85}
\renewcommand{\textfraction}{0.15}
\renewcommand{\floatpagefraction}{0.8}

%\RequirePackage[section]{placeins}

%% SECTION TITLES
\RequirePackage[compact,sf,bf]{titlesec}
\titleformat{\section}[block]
  {\fontsize{15}{17}\bfseries\sffamily}
  {\thesection}
  {0.4em}{}
\titleformat{\subsection}[block]
  {\fontsize{12}{14}\bfseries\sffamily}
  {\thesubsection}
  {0.4em}{}
\titlespacing{\section}{0pt}{*5}{*1}
\titlespacing{\subsection}{0pt}{*2}{*0.2}


%% TITLE PAGE
\def\Date{\number\day}
\def\Month{\ifcase\month\or
 January\or February\or March\or April\or May\or June\or
 July\or August\or September\or October\or November\or December\fi}
\def\Year{\number\year}

\makeatletter
\def\wp#1{\gdef\@wp{#1}}\def\@wp{??/??}
\def\jel#1{\gdef\@jel{#1}}\def\@jel{??}
\def\showjel{{\large\textsf{\textbf{JEL classification:}}~\@jel}}
\def\nojel{\def\showjel{}}
\def\addresses#1{\gdef\@addresses{#1}}\def\@addresses{??}
\def\cover{{\sffamily\setcounter{page}{0}
        \thispagestyle{empty}
        \placefig{2}{1.5}{width=5cm}{monash2}
        \placefig{16.9}{1.5}{width=2.1cm}{MBusSchool}
        \begin{textblock}{4}(16.9,4)ISSN 1440-771X\end{textblock}
        \begin{textblock}{7}(12.7,27.9)\hfill
        \includegraphics[height=0.7cm]{AACSB}~~~
        \includegraphics[height=0.7cm]{EQUIS}~~~
        \includegraphics[height=0.7cm]{AMBA}
        \end{textblock}
        \vspace*{2cm}
        \begin{center}\Large
        Department of Econometrics and Business Statistics\\[.5cm]
        \footnotesize http://monash.edu/business/ebs/research/publications
        \end{center}\vspace{2cm}
        \begin{center}
        \fbox{\parbox{14cm}{\begin{onehalfspace}\centering\Huge\vspace*{0.3cm}
                \textsf{\textbf{\expandafter{\@title}}}\vspace{1cm}\par
                \LARGE\@author\end{onehalfspace}
        }}
        \end{center}
        \vfill
                \begin{center}\Large
                \Month~\Year\\[1cm]
                Working Paper \@wp
        \end{center}\vspace*{2cm}}}
\def\pageone{{\sffamily\setstretch{1}%
        \thispagestyle{empty}%
        \vbox to \textheight{%
        \raggedright\baselineskip=1.2cm
     {\fontsize{24.88}{30}\sffamily\textbf{\expandafter{\@title}}}
        \vspace{2cm}\par
        \hspace{1cm}\parbox{14cm}{\sffamily\large\@addresses}\vspace{1cm}\vfill
        \hspace{1cm}{\large\Date~\Month~\Year}\\[1cm]
        \hspace{1cm}\showjel\vss}}}
\def\blindtitle{{\sffamily
     \thispagestyle{plain}\raggedright\baselineskip=1.2cm
     {\fontsize{24.88}{30}\sffamily\textbf{\expandafter{\@title}}}\vspace{1cm}\par
        }}
\def\titlepage{{\cover\newpage\pageone\newpage\blindtitle}}

\def\blind{\def\titlepage{{\blindtitle}}\let\maketitle\blindtitle}
\def\titlepageonly{\def\titlepage{{\pageone\end{document}}}}
\def\nocover{\def\titlepage{{\pageone\newpage\blindtitle}}\let\maketitle\titlepage}
\let\maketitle\titlepage
\makeatother

%% SPACING
\RequirePackage{setspace}
\spacing{1.5}

%% LINE AND PAGE BREAKING
\sloppy
\clubpenalty = 10000
\widowpenalty = 10000
\brokenpenalty = 10000
\RequirePackage{microtype}

%% PARAGRAPH BREAKS
\setlength{\parskip}{1.4ex}
\setlength{\parindent}{0em}

%% HYPERLINKS
\RequirePackage{xcolor} % Needed for links
\definecolor{darkblue}{rgb}{0,0,.6}
\RequirePackage{url}

\makeatletter
\@ifpackageloaded{hyperref}{}{\RequirePackage{hyperref}}
\makeatother
\hypersetup{
     citecolor=0 0 0,
     breaklinks=true,
     bookmarksopen=true,
     bookmarksnumbered=true,
     linkcolor=darkblue,
     urlcolor=blue,
     citecolor=darkblue,
     colorlinks=true}

%% KEYWORDS
\newenvironment{keywords}{\par\vspace{0.5cm}\noindent{\sffamily\textbf{Keywords:}}}{\vspace{0.25cm}\par\hrule\vspace{0.5cm}\par}

%% ABSTRACT
\renewenvironment{abstract}{\begin{minipage}{\textwidth}\parskip=1.4ex\noindent
\hrule\vspace{0.1cm}\par{\sffamily\textbf{\abstractname}}\newline}
  {\end{minipage}}


\usepackage[T1]{fontenc}
\usepackage[utf8]{inputenc}

\usepackage[showonlyrefs]{mathtools}
\usepackage[no-weekday]{eukdate}

%% BIBLIOGRAPHY

\makeatletter
\@ifpackageloaded{biblatex}{}{\usepackage[style=authoryear-comp, backend=biber, natbib=true]{biblatex}}
\makeatother
\ExecuteBibliographyOptions{bibencoding=utf8,minnames=1,maxnames=3, maxbibnames=99,dashed=false,terseinits=true,giveninits=true,uniquename=false,uniquelist=false,doi=false, isbn=false,url=true,sortcites=false}

\DeclareFieldFormat{url}{\texttt{\url{#1}}}
\DeclareFieldFormat[article]{pages}{#1}
\DeclareFieldFormat[inproceedings]{pages}{\lowercase{pp.}#1}
\DeclareFieldFormat[incollection]{pages}{\lowercase{pp.}#1}
\DeclareFieldFormat[article]{volume}{\mkbibbold{#1}}
\DeclareFieldFormat[article]{number}{\mkbibparens{#1}}
\DeclareFieldFormat[article]{title}{\MakeCapital{#1}}
\DeclareFieldFormat[inproceedings]{title}{#1}
\DeclareFieldFormat{shorthandwidth}{#1}
% No dot before number of articles
\usepackage{xpatch}
\xpatchbibmacro{volume+number+eid}{\setunit*{\adddot}}{}{}{}
% Remove In: for an article.
\renewbibmacro{in:}{%
  \ifentrytype{article}{}{%
  \printtext{\bibstring{in}\intitlepunct}}}

\makeatletter
\DeclareDelimFormat[cbx@textcite]{nameyeardelim}{\addspace}
\makeatother
\renewcommand*{\finalnamedelim}{%
  %\ifnumgreater{\value{liststop}}{2}{\finalandcomma}{}% there really should be no funny Oxford comma business here
  \addspace\&\space}


\nojel

\RequirePackage[absolute,overlay]{textpos}
\setlength{\TPHorizModule}{1cm}
\setlength{\TPVertModule}{1cm}
\def\placefig#1#2#3#4{\begin{textblock}{.1}(#1,#2)\rlap{\includegraphics[#3]{#4}}\end{textblock}}


\nocover

\author{Talagala~Thiyanga}
\addresses{\textbf{Talagala Thiyanga}\newline
University of Sri Jayewardenepura
\newline{Email: \href{mailto:ttalagala@sjp.ac.lk}{\nolinkurl{ttalagala@sjp.ac.lk}}}\\[1cm]
}

\date{\sf\Date~\Month~\Year}
\makeatletter
 \lfoot{\sf Thiyanga: \@date}
\makeatother

%% Any special functions or other packages can be loaded here.


\begin{document}
\maketitle

{
\setcounter{tocdepth}{2}
\tableofcontents
}
\newpage

\hypertarget{introduction}{%
\section{Introduction}\label{introduction}}

\hypertarget{models-for-stationary-time-series}{%
\subsection{Models for stationary time series}\label{models-for-stationary-time-series}}

\hypertarget{models-for-nonstationary-time-series}{%
\subsection{Models for nonstationary time series}\label{models-for-nonstationary-time-series}}

First, we will look at the theoretical properties of these models.

\hypertarget{autoregressive-process}{%
\section{Autoregressive process}\label{autoregressive-process}}

\hypertarget{properties-of-ar1-model}{%
\subsection{Properties of AR(1) model}\label{properties-of-ar1-model}}

Consider the following \(AR(1)\) model.

\begin{equation}
  \label{eq:1}
Y_t=\phi_0+\phi_1Y_{t-1}+\epsilon_{t}
\end{equation}

where \({\epsilon_t}\) is assumed to be a white noise process with mean zero and variance \(\sigma^2\).

\hypertarget{mean}{%
\subsubsection{Mean}\label{mean}}

Assuming that the series is weak stationary, we have \(E(Y_t)=\mu\), \(Var(Y_t)=\gamma_0\), and \(Cov(Y_t, Y_{t-k})=\gamma_k\), where \(\mu\) and \(\gamma_0\) are constants. Given that \({\epsilon_t}\) is a white noise, we have \(E(\epsilon_t)=0\). The mean of \(AR(1)\) process can be computed as follows:

\[
\begin{aligned}
  E(Y_t) &= E(\phi_0+\phi_1 Y_{t-1}) \\
         &= E(\phi_0) +E(\phi_1 Y_{t-1}) \\
         &= \phi_0 +\phi_1 E(Y_{t-1}). \\
\end{aligned}
\]

Under the stationarity condition, \(E(Y_t)=E(Y_{t-1})=\mu\). Thus we get

\[\mu = \phi_0+\phi_1\mu.\]

Solving for \(\mu\) yields

\begin{equation}
  \label{eq:2}
E(Y_t)=\mu=\frac{\phi_0}{1-\phi_1}.
\end{equation}

The results has two constraints for \(Y_t\). First, the mean of \(Y_t\) exists if \(\phi \neq 1 .\) The mean of \(Y_t\) is zero if and only if \(\phi_0=0\).

\hypertarget{variance-and-stationary-condition-of-ar-1-process}{%
\subsubsection{Variance and stationary condition of AR (1) process}\label{variance-and-stationary-condition-of-ar-1-process}}

First take variance of both sides of Eq. (1)

\[Var(Y_t)=Var(\phi_0+\phi_1 Y_{t-1}+\epsilon_t)\]

The \(Y_{t-1}\) occurred before time \(t\). The \(\epsilon_t\) does not depend on any past observation. Hence, \(cov(Y_{t-1}, \epsilon_t)= 0\). Furthermore, \({\epsilon_t}\) is a white noise and hence

\[Var(Y_t)=\phi_1^2 Var(Y_{t-1})+\sigma^2.\]

Under the stationarity condition, \(Var(Y_t)=Var(Y_{t-1})\), so that,

\[Var(Y_t)=\frac{\sigma^2}{1-\phi_1^2}.\]

provided that \(\phi_1^2 < 1\) or \(|\phi| < 1\) (The variance of a random variable is bounded and non-negative). The necessary and sufficient condition for the \(AR(1)\) model in Eq. (1) to be weakly stationary is \(|\phi| < 1\). This condition is equivalent to saying that the root of \(1-\phi_1B = 0\) must lie outside the unit circle. This can be explained as below

Using the backshift notation we can write \(AR(1)\) process as

\[Y_t = \phi_0 + \phi_1BY_{t} + \epsilon_t.\]

Then we get

\[(1-\phi_1B)Y_t=\phi_0 + \epsilon_t.\] The \(AR(1)\) process is said to be stationary if the roots of \((1-\phi_1B)=0\) lie outside the unit circle.

\hypertarget{covariance}{%
\subsection{Covariance}\label{covariance}}

The covariance \(\gamma_k=Cov(Y_t, Y_{t-k})\) is called the lag-\(k\) autocovariance of \(Y_t\). The two main properties of \(\gamma_k\): (a) \(\gamma_0=Var(Y_t)\) and (b) \(\gamma_{-k}=\gamma_{k}\).

The lag-\(k\) autocovariance of \(Y_t\) is

\begin{equation}
  \label{eq:3}
\begin{aligned}
  \gamma_k &= Cov(Y_t, Y_{t-k}) \\
         &= E[(Y_t-\mu)(Y_{t-k}-\mu)] \\
         &= E[Y_tY_{t-k}-Y_t\mu-\mu Y_{t-k} +\mu^2] \\
         &= E(Y_t Y_{t-k}) - \mu^2. \\
\end{aligned}
\end{equation}

Now we have

\begin{equation}
  \label{eq:3}
  E(Y_t Y_{t-k}) = \gamma_k + \mu^2
\end{equation}

\hypertarget{autocorrelation-function-of-an-ar1-process}{%
\subsubsection{Autocorrelation function of an AR(1) process}\label{autocorrelation-function-of-an-ar1-process}}

To derive autocorrelation function of an AR(1) process we first multiply both sides of Eq. (1) by \(Y_{t-k}\) and take expected values:

\[E(Y_tY_{t-k})=\phi_0E(Y_{t-k})+\phi_1 E(Y_{t-1}Y_{t-k})+E(\epsilon_tY_{t-k})\]
Since \(\epsilon_t\) and \(Y_{t-k}\) are independent and using the results in Eq. (4)

\[\gamma_k + \mu^2 = \phi_0 \mu+\phi_1(\gamma_{k-1}+\mu^2)\]

Substituting the results in Eq. (2) to Eq. (4) we get

\begin{equation}
\label{eq:5}
\gamma_k = \phi_1 \gamma_{k-1}.
\end{equation}

The autocorrelation function is defined as

\[\rho_k = \frac{\gamma_k}{\gamma_0}\].

Setting \(k=1\), we get \(\gamma_1 = \phi_1\gamma_0.\) Hence,

\[\rho_1=\phi_1.\]

Similarly with \(k=2\), \(\gamma_2 = \phi_1 \gamma_1\). Dividing both sides by \(\gamma_0\) and substituting with \(\rho_1=\phi_1\) we get

\[\rho_2=\phi_1^2.\]

Now it is easy to see that in general

\[\rho_k = \frac{\gamma_k}{\gamma_0}=\phi_1^k \]

for \(k=0, 1, 2, 3, ...\).

Since \(|\phi_1| < 1,\) the autocorrelation function is an exponentially decreasing as the number of lags \(k\) increases. There are two features in the ACF of AR(1) process depending on the sign of \(\phi_1\). They are,

\begin{enumerate}
\def\labelenumi{\arabic{enumi}.}
\item
  If \(0 < \phi_1 < 1,\) all correlations are positive.
\item
  if \(-1 < \phi_1 < 0,\) the lag 1 autocorrelation is negative (\(\phi_1=\phi_1\)) and the signs of successive autocorrelations alternate from positive to negative with their magnitudes decreasing exponentially.
\end{enumerate}

\hypertarget{properties-of-ar2-model}{%
\subsection{Properties of AR(2) model}\label{properties-of-ar2-model}}

Now consider an second-order autoregressive process (AR(2))

\begin{equation}
  \label{eq:1}
Y_t=\phi_0+\phi_1Y_{t-1}+\phi_2Y_{t-2}+\epsilon_t.
\end{equation}

\hypertarget{mean-1}{%
\subsubsection{Mean}\label{mean-1}}

\textbf{Question 1:} Using the same technique as that of the AR(1), show that

\[E(Y_t) = \mu = \frac{\phi_0}{1-\phi_1 - \phi_2}\] and the mean of \(Y_t\) exists if \(\phi_1 + \phi_2 \neq 1\).

\hypertarget{variance}{%
\subsubsection{Variance}\label{variance}}

\textbf{Question 2:} Show that \[Var(Y_t) = \frac{(1-\phi_2)\sigma^2}{(1+\phi_2)((1+\phi_2)^2-\phi_1^2)}.\]

Here is a guide to the solution

Start with

\[Var(Y_t)=Var(\phi_0+\phi_1Y_{t-1}+\phi_2Y_{t-2}+\epsilon_t)\]

Solve it until you obtain the Eq. (a) as shown below.

\begin{equation}
\tag{a}
\gamma_0 (1-\alpha_1^2 - \alpha_1^2) = 2\alpha_1\alpha_2\gamma_1+\sigma^2.
\end{equation}

Next multiply both sides of Eq. (6) by \(Y_{t-1}\) and obtain a expression for \(\gamma_1\). Let's call this Eq. (b).

Solve Eq. (a) and (b) for \(\gamma_0.\)

\hypertarget{stationarity-of-ar2-process}{%
\subsubsection{Stationarity of AR(2) process}\label{stationarity-of-ar2-process}}

To discuss the stationarity condition of the \(AR(2)\) process we use the roots of the characteristic polynomial. Here is the illustration.

Using the backshift notation we can write \(AR(2)\) process as

\[Y_t = \phi_0 + \phi_1 BY_{t} + \phi_2 B^2 Y_{t} + \epsilon_t.\]

Furthermore, we get

\[(1-\phi_1 B - \phi_2 B^2) Y_t = \phi_0 + \epsilon_t.\]

The \textbf{characteristic polynomial} of \(AR(2)\) process is

\[\Phi(B)=1-\phi_1 B - \phi_2 B^2.\]

and the corresponding \textbf{AR characteristic equation}

\[1-\phi_1 B - \phi_2 B^2=0.\]

For stationarity, the roots of AR characteristic equation must lie outside the unit circle. The two roots of the AR characteristic equation are

\[\frac{\phi_1 }{-2\phi_2}\]

Using algebraic manipulation, we can show that these roots will exceed 1 in modulus if and only if simultaneously \(\phi_1 + \phi_2 < 1,\) \(\phi_2-\phi_1 < 1,\) and \(|\phi_2| < 1.\) This is called the stationarity condition of \(AR(2)\) process.

\hypertarget{autocorrelation-function-of-an-ar2-process}{%
\subsubsection{Autocorrelation function of an AR(2) process}\label{autocorrelation-function-of-an-ar2-process}}

To derive autocorrelation function of an AR(2) process we first multiply both sides of Eq. (6) by \(Y_{t-k}\) and take expected values:

\begin{align}
E(Y_tY_{t-k}) &= E(\phi_0Y_{t-k}+\alpha_1Y_{t-1}Y_{t-k}+\alpha_2Y_{t-2}Y_{t-k})+\epsilon_tY_{t-k} \\
&= \phi_0 E(Y_{t-k})+\phi_{1}E(Y_{t-1}Y_{t-k}) + \phi_2 E(X_{t-2} X_{t-k}) + E(\epsilon_tX_{t-k}).
\end{align}

Using the independence between \(\epsilon_t\) and \(Y_{t-1}\), \(E(\epsilon_t X_{t-k})=0\) and the results in Eq. 3 (This valid for AR(2)) we have

\[\gamma_k + \mu^2 = \gamma_0 \mu + \alpha_1 (\gamma_{k-1}+\mu^2)+\phi_2 (\gamma_{k-2}+\mu^2).\]

(Note that \(E(X_{t-1}X_{t-k})=E(X_{t-1}X_{(t-1)-(k-1)}=\gamma_{k-1})\))

Solving for \(\gamma_k\) we get

\begin{align}
 \gamma_k=\phi_1\gamma_{k-1}+\phi_2\gamma_{k-2}.
\end{align}

By dividing the both sides of Eq. (9) by \(\gamma_0\), we have

\begin{align}
 \rho_k=\phi_1\rho_{k-1}+\phi_2\rho_{k-2}.
\end{align}

for \(k>0\).

Setting \(k=1\) and using \(\rho_0=1\) and \(\rho_{-1}=\rho_1\), we get \textbf{the Yule-Walker equation for \(AR(2)\) process.}

\[\rho_1=\phi_1+\phi_2 \rho_1\] or

\[\rho_1 = \frac{\phi_1}{1-\phi_2}\].

Similarly, we can show that

\[\rho_2 = \frac{\phi_2(1-\phi_2)+\phi_1^2}{(1-\phi_2)}.\]

\hypertarget{properties-of-arp-model}{%
\subsection{Properties of AR(p) model}\label{properties-of-arp-model}}

The \(p\)th order autoregressive model can be written as

\begin{align}
Y_t = \phi_0 + \phi_1Y_{t-1}+\phi_2 Y_{t-2}+ ... + \epsilon_t.
\end{align}

The AR characteristic equation is

\[1-\phi_1B-\phi_2B^2-...-\phi_pB^p=0.\]

For stationarity of \(AR(p)\) process, the \(p\) roots of the AR characteristic must lie outside the unit circle.

\hypertarget{mean-2}{%
\subsubsection{Mean}\label{mean-2}}

\textbf{Question: } Find \(E(Y_t)\) of \(AR(p)\) process.

\hypertarget{variance-1}{%
\subsubsection{Variance}\label{variance-1}}

\textbf{Question: } Find \(Var(Y_t)\) of \(AR(p)\) process.

\hypertarget{autocorrelation-function-of-an-arp-process}{%
\subsubsection{Autocorrelation function of an AR(p) process}\label{autocorrelation-function-of-an-arp-process}}

\textbf{Question: } Assuming stationarity and multiplying both sides of Eq. (11) obtain the following recursive relationship.

\begin{align}
\rho_k = \phi_1\rho_{k-1}+\phi_2 \rho_{k-2} + ... + \phi_p \rho_{k-p}.
\end{align}

Setting \(k=1, 2, ..., p\) into Eq. (11) and using \(\rho_0=1\) and \(\rho_{-k}=\rho_k\), we get the Yule-Walker equations for \(AR(p)\) process

\begin{equation}
  \label{eq:13}
\begin{aligned}
  \rho_1 &= \phi_1+\phi_2 \rho_{1} + ... + \phi_p \rho_{p-1}\\
  \rho_2 &= \phi_1 \rho_1+\phi_2  + ... + \phi_p \rho_{p-2}\\
  ... \\
  \rho_p &= \phi_1 \rho_{p-1} +\phi_2 \rho_{p-2}  + ... + \phi_p \\
\end{aligned}
\end{equation}

\hypertarget{moving-average-ma-models}{%
\section{Moving average (MA) models}\label{moving-average-ma-models}}

We first derive the properties of \(MA(1)\) and \(MA(2)\) models and then give the results for the general \(MA(q)\) model.

\hypertarget{properties-of-ma1-model}{%
\subsection{Properties of MA(1) model}\label{properties-of-ma1-model}}

\hypertarget{mean-3}{%
\subsubsection{Mean}\label{mean-3}}

\hypertarget{variance-2}{%
\subsubsection{Variance}\label{variance-2}}

\hypertarget{autocorrelation-function-of-an-ma1-process}{%
\subsubsection{Autocorrelation function of an MA(1) process}\label{autocorrelation-function-of-an-ma1-process}}

\hypertarget{partial-autocorrelation-function-of-an-ma1-process}{%
\subsubsection{Partial autocorrelation function of an MA(1) process}\label{partial-autocorrelation-function-of-an-ma1-process}}

\hypertarget{properties-of-ma2-model}{%
\subsection{Properties of MA(2) model}\label{properties-of-ma2-model}}

\hypertarget{mean-4}{%
\subsubsection{Mean}\label{mean-4}}

\hypertarget{variance-3}{%
\subsubsection{Variance}\label{variance-3}}

\hypertarget{autocorrelation-function-of-an-ma2-process}{%
\subsubsection{Autocorrelation function of an MA(2) process}\label{autocorrelation-function-of-an-ma2-process}}

\hypertarget{partial-autocorrelation-function-of-an-ma2-process}{%
\subsubsection{Partial autocorrelation function of an MA(2) process}\label{partial-autocorrelation-function-of-an-ma2-process}}

\hypertarget{properties-of-map-model}{%
\subsection{Properties of MA(p) model}\label{properties-of-map-model}}

\hypertarget{mean-5}{%
\subsubsection{Mean}\label{mean-5}}

\hypertarget{variance-4}{%
\subsubsection{Variance}\label{variance-4}}

\hypertarget{autocorrelation-function-of-an-maq-process}{%
\subsubsection{Autocorrelation function of an MA(q) process}\label{autocorrelation-function-of-an-maq-process}}

\hypertarget{partial-autocorrelation-function-of-an-maq-process}{%
\subsubsection{Partial autocorrelation function of an MA(q) process}\label{partial-autocorrelation-function-of-an-maq-process}}

\hypertarget{autoregressive-and-moving-average-arma-models}{%
\section{Autoregressive and Moving-average (ARMA) models}\label{autoregressive-and-moving-average-arma-models}}

\hypertarget{unit-root-nonstationarity}{%
\section{Unit root nonstationarity}\label{unit-root-nonstationarity}}

\hypertarget{arima-models}{%
\section{ARIMA models}\label{arima-models}}

\hypertarget{sarima-models}{%
\section{SARIMA models}\label{sarima-models}}

\hypertarget{forecasting-model-building-process-using-r}{%
\section{Forecasting model building process using R}\label{forecasting-model-building-process-using-r}}

\hypertarget{appendix}{%
\section{Appendix}\label{appendix}}

Some important rules in statistics

\[Var(X+Y) = Var(X) + Var(Y) + 2Cov(X+Y)\]

\printbibliography

\end{document}
